% Latent Gaussian Models

\subsection{Model Structure} \label{subsubsec:model-structure}
Latent Gaussian models constitute a class of Bayesian hierarchical models characterized by a three-level structure:

\begin{enumerate}
    \item \textbf{Data level}: The observations $\mathbf{y} = (y_1, \ldots, y_n)^T$ are modeled conditionally on latent parameters $\boldsymbol{\eta}$ and hyperparameters $\boldsymbol{\theta}$, with density $\pi(\mathbf{y}|\boldsymbol{\eta}, \boldsymbol{\theta})$.

    \item \textbf{Latent level}: The latent parameters $\boldsymbol{\eta}$ follow a Gaussian distribution conditional on hyperparameters $\boldsymbol{\theta}$:
    \begin{equation}
        \pi(\boldsymbol{\eta}|\boldsymbol{\theta}) = \mathcal{N}(\boldsymbol{\eta}|\boldsymbol{\mu}(\boldsymbol{\theta}), \mathbf{\Sigma}(\boldsymbol{\theta})).
    \end{equation}

    \item \textbf{Hyperparameter level}: The hyperparameters $\boldsymbol{\theta}$ are assigned a prior density $\pi(\boldsymbol{\theta})$.
\end{enumerate}

The defining characteristic of LGMs is the Gaussian prior on the latent field $\boldsymbol{\eta}$. For computational efficiency, the latent field is often structured as a Gaussian Markov Random Field (GMRF) with a sparse precision matrix $\mathbf{Q}_{\boldsymbol{\eta}}(\boldsymbol{\theta}) = \mathbf{\Sigma}^{-1}(\boldsymbol{\theta})$, facilitating efficient inference even in high dimensions.

In standard LGMs, as described by \cite{rue2009approximate}, a univariate link function connects a parameter of interest at the data level to a structured additive predictor at the latent level:

\begin{equation}
    g(\mu_i) = \eta_i,
\end{equation}

where $\mu_i = \mathrm{E}(y_i|\eta_i, \boldsymbol{\theta})$ or $\mu_i = Q_p(y_i|\eta_i, \boldsymbol{\theta})$ for some quantile function $Q_p$, and $g(\cdot)$ is the link function. The structured additive predictor $\eta_i$ typically takes the form:

\begin{equation}
    \eta_i = \beta_0 + \sum_{k=1}^K \beta_k z_{i,k} + \sum_{j=1}^J u_j a_{i,j} + \epsilon_i,
\end{equation}

where $\beta_0$ is an intercept, $\{\beta_k\}$ are linear fixed effects, $\{u_j\}$ are random effects with specified dependency structures, and $\{\epsilon_i\}$ are unstructured model errors.

\subsubsection{Extended LGMs with Multivariate Link Functions} \label{subsubsec:ext-lgm}
Many applications require modeling multiple parameters of the data distribution simultaneously, motivating the development of extended LGMs with multivariate link functions \citep{geirsson2020lgm}. In these models, multiple parameters of the data distribution are linked to separate latent Gaussian fields, providing greater flexibility in capturing complex data structures.

For extended LGMs, we assume that the data can be grouped according to some structure (e.g., spatial locations, time points, or categories), with each group $i$ having $n_i$ observations. Let the total number of groups be $G$. For group $i$, the parameters are denoted by $\boldsymbol{\psi}_i = (\psi_{1,i}, \psi_{2,i}, \ldots, \psi_{M,i})^T \in D$, where $D$ is a subspace of $\mathbb{R}^M$ and $M$ is the number of parameters. The joint density for all observations is:

\begin{equation}
    \pi(\mathbf{y}|\boldsymbol{\psi}_1, \boldsymbol{\psi}_2, \ldots, \boldsymbol{\psi}_G) = \prod_{i=1}^G \pi(\mathbf{y}_i|\boldsymbol{\psi}_i) = \prod_{i=1}^G \prod_{j \in A_i} \pi(y_{i,j}|\boldsymbol{\psi}_i),
\end{equation}

where $A_i$ is the index set for group $i$.

We define an $M$-variate link function $\mathbf{g}: D \rightarrow \mathbb{R}^M$ such that:

\begin{equation}
    \mathbf{g}(\boldsymbol{\psi}_i) = (\eta_{1,i}, \eta_{2,i}, \ldots, \eta_{M,i})^T = \boldsymbol{\eta}_i \in \mathbb{R}^M.
\end{equation}

For each parameter type $m \in \{1, \ldots, M\}$, we specify a linear model for the latent parameters $\boldsymbol{\eta}_m = (\eta_{m,1}, \ldots, \eta_{m,G})^T$:

\begin{equation}
    \boldsymbol{\eta}_m = \mathbf{X}_m \boldsymbol{\beta}_m + \mathbf{A}_m \mathbf{u}_m + \boldsymbol{\epsilon}_m,
\end{equation}

where $\boldsymbol{\beta}_m$ represents fixed effects, $\mathbf{X}_m$ is the corresponding design matrix, $\mathbf{u}_m$ represents random effects, $\mathbf{A}_m$ is the corresponding weight matrix, and $\boldsymbol{\epsilon}_m$ represents unstructured error terms. The terms $\boldsymbol{\beta}_m$, $\mathbf{u}_m$, and $\boldsymbol{\epsilon}_m$ are assigned Gaussian prior densities and are assumed to be a priori independent.
